% Options for packages loaded elsewhere
% Options for packages loaded elsewhere
\PassOptionsToPackage{unicode}{hyperref}
\PassOptionsToPackage{hyphens}{url}
\PassOptionsToPackage{dvipsnames,svgnames,x11names}{xcolor}
%
\documentclass[
  letterpaper,
  DIV=11,
  numbers=noendperiod]{scrartcl}
\usepackage{xcolor}
\usepackage{amsmath,amssymb}
\setcounter{secnumdepth}{5}
\usepackage{iftex}
\ifPDFTeX
  \usepackage[T1]{fontenc}
  \usepackage[utf8]{inputenc}
  \usepackage{textcomp} % provide euro and other symbols
\else % if luatex or xetex
  \usepackage{unicode-math} % this also loads fontspec
  \defaultfontfeatures{Scale=MatchLowercase}
  \defaultfontfeatures[\rmfamily]{Ligatures=TeX,Scale=1}
\fi
\usepackage{lmodern}
\ifPDFTeX\else
  % xetex/luatex font selection
\fi
% Use upquote if available, for straight quotes in verbatim environments
\IfFileExists{upquote.sty}{\usepackage{upquote}}{}
\IfFileExists{microtype.sty}{% use microtype if available
  \usepackage[]{microtype}
  \UseMicrotypeSet[protrusion]{basicmath} % disable protrusion for tt fonts
}{}
\makeatletter
\@ifundefined{KOMAClassName}{% if non-KOMA class
  \IfFileExists{parskip.sty}{%
    \usepackage{parskip}
  }{% else
    \setlength{\parindent}{0pt}
    \setlength{\parskip}{6pt plus 2pt minus 1pt}}
}{% if KOMA class
  \KOMAoptions{parskip=half}}
\makeatother
% Make \paragraph and \subparagraph free-standing
\makeatletter
\ifx\paragraph\undefined\else
  \let\oldparagraph\paragraph
  \renewcommand{\paragraph}{
    \@ifstar
      \xxxParagraphStar
      \xxxParagraphNoStar
  }
  \newcommand{\xxxParagraphStar}[1]{\oldparagraph*{#1}\mbox{}}
  \newcommand{\xxxParagraphNoStar}[1]{\oldparagraph{#1}\mbox{}}
\fi
\ifx\subparagraph\undefined\else
  \let\oldsubparagraph\subparagraph
  \renewcommand{\subparagraph}{
    \@ifstar
      \xxxSubParagraphStar
      \xxxSubParagraphNoStar
  }
  \newcommand{\xxxSubParagraphStar}[1]{\oldsubparagraph*{#1}\mbox{}}
  \newcommand{\xxxSubParagraphNoStar}[1]{\oldsubparagraph{#1}\mbox{}}
\fi
\makeatother


\usepackage{longtable,booktabs,array}
\usepackage{calc} % for calculating minipage widths
% Correct order of tables after \paragraph or \subparagraph
\usepackage{etoolbox}
\makeatletter
\patchcmd\longtable{\par}{\if@noskipsec\mbox{}\fi\par}{}{}
\makeatother
% Allow footnotes in longtable head/foot
\IfFileExists{footnotehyper.sty}{\usepackage{footnotehyper}}{\usepackage{footnote}}
\makesavenoteenv{longtable}
\usepackage{graphicx}
\makeatletter
\newsavebox\pandoc@box
\newcommand*\pandocbounded[1]{% scales image to fit in text height/width
  \sbox\pandoc@box{#1}%
  \Gscale@div\@tempa{\textheight}{\dimexpr\ht\pandoc@box+\dp\pandoc@box\relax}%
  \Gscale@div\@tempb{\linewidth}{\wd\pandoc@box}%
  \ifdim\@tempb\p@<\@tempa\p@\let\@tempa\@tempb\fi% select the smaller of both
  \ifdim\@tempa\p@<\p@\scalebox{\@tempa}{\usebox\pandoc@box}%
  \else\usebox{\pandoc@box}%
  \fi%
}
% Set default figure placement to htbp
\def\fps@figure{htbp}
\makeatother





\setlength{\emergencystretch}{3em} % prevent overfull lines

\providecommand{\tightlist}{%
  \setlength{\itemsep}{0pt}\setlength{\parskip}{0pt}}



 


\KOMAoption{captions}{tableheading}
\makeatletter
\@ifpackageloaded{caption}{}{\usepackage{caption}}
\AtBeginDocument{%
\ifdefined\contentsname
  \renewcommand*\contentsname{Table of contents}
\else
  \newcommand\contentsname{Table of contents}
\fi
\ifdefined\listfigurename
  \renewcommand*\listfigurename{List of Figures}
\else
  \newcommand\listfigurename{List of Figures}
\fi
\ifdefined\listtablename
  \renewcommand*\listtablename{List of Tables}
\else
  \newcommand\listtablename{List of Tables}
\fi
\ifdefined\figurename
  \renewcommand*\figurename{Figure}
\else
  \newcommand\figurename{Figure}
\fi
\ifdefined\tablename
  \renewcommand*\tablename{Table}
\else
  \newcommand\tablename{Table}
\fi
}
\@ifpackageloaded{float}{}{\usepackage{float}}
\floatstyle{ruled}
\@ifundefined{c@chapter}{\newfloat{codelisting}{h}{lop}}{\newfloat{codelisting}{h}{lop}[chapter]}
\floatname{codelisting}{Listing}
\newcommand*\listoflistings{\listof{codelisting}{List of Listings}}
\makeatother
\makeatletter
\makeatother
\makeatletter
\@ifpackageloaded{caption}{}{\usepackage{caption}}
\@ifpackageloaded{subcaption}{}{\usepackage{subcaption}}
\makeatother
\usepackage{bookmark}
\IfFileExists{xurl.sty}{\usepackage{xurl}}{} % add URL line breaks if available
\urlstyle{same}
\hypersetup{
  pdftitle={Integrated KSAO Framework for Substance Use Disorder Counselors},
  pdfauthor={KSAO Workforce Development Project},
  colorlinks=true,
  linkcolor={blue},
  filecolor={Maroon},
  citecolor={Blue},
  urlcolor={Blue},
  pdfcreator={LaTeX via pandoc}}


\title{Integrated KSAO Framework for Substance Use Disorder Counselors}
\usepackage{etoolbox}
\makeatletter
\providecommand{\subtitle}[1]{% add subtitle to \maketitle
  \apptocmd{\@title}{\par {\large #1 \par}}{}{}
}
\makeatother
\subtitle{A Comprehensive Competency Map Derived from CASAC Textbook
Analysis}
\author{KSAO Workforce Development Project}
\date{2025-05-17}
\begin{document}
\maketitle

\renewcommand*\contentsname{Table of contents}
{
\hypersetup{linkcolor=}
\setcounter{tocdepth}{3}
\tableofcontents
}

\section{Introduction}\label{introduction}

This document presents the integrated Knowledge, Skills, Abilities, and
Other Characteristics (KSAOs) framework for substance use disorder
counselors, derived from a comprehensive analysis of the CASAC
(Credentialed Alcoholism and Substance Abuse Counselor) certification
textbook. The analysis focused on Chapters 1-2 and Appendices, using
Google's Gemini 2.5 Pro model to identify and categorize competencies.

This framework is designed to serve as a foundation for evidence-based
workforce development in addiction counseling, moving beyond traditional
certification requirements to focus on actual competencies required for
effective practice.

\section{Methodology}\label{methodology}

\subsection{Data Sources}\label{data-sources}

The analysis was conducted on the following chapters from the CASAC
textbook:

\begin{enumerate}
\def\labelenumi{\arabic{enumi}.}
\item
  \textbf{Chapter 1}: Covering SUD terminology, neurobiology, models of
  addiction, risk factors, stages of addiction, common substances,
  recovery principles, and the Stages of Change model.
\item
  \textbf{Chapter 2}: Focusing on systems of care, telehealth, harm
  reduction, trauma-informed systems, social determinants of health, and
  related topics.
\item
  \textbf{Appendices}: Including commonly used drugs, mental health
  conditions, medications, and IC\&RC domains and job tasks.
\end{enumerate}

\subsection{Analysis Process}\label{analysis-process}

The analysis followed a rigorous five-step process:

\begin{enumerate}
\def\labelenumi{\arabic{enumi}.}
\item
  \textbf{OCR Text Extraction}: Used Gemini 2.5 Pro's multimodal
  capabilities to extract text from TIFF images of textbook pages.
\item
  \textbf{Chapter-Level KSAO Analysis}: Each chapter was analyzed
  individually to identify KSAOs, with documentation of the thinking
  process.
\item
  \textbf{KSAO Integration}: Individual chapter analyses were integrated
  into a comprehensive framework, resolving redundancies and
  inconsistencies.
\end{enumerate}

For each KSAO, the analysis identified: - Clear name/title - Complete
description - Classification (Knowledge, Skill, Ability, or Other
characteristic) - Specificity level (general or specialized) - Related
O*NET occupational categories - Stability/malleability (fixed
vs.~developable) - Explicit/tacit orientation (explicitly taught
vs.~tacitly acquired) - Prerequisites or developmental relationships

\section{Integrated KSAO Framework}\label{integrated-ksao-framework}

\subsection{THE COMPREHENSIVE INTEGRATED KSAO FRAMEWORK FOR SUD
COUNSELORS}\label{the-comprehensive-integrated-ksao-framework-for-sud-counselors}

This framework integrates KSAOs from the ``Appendices Analysis (AA),''
``Chapter 2 Analysis (AC2),'' and ``Textbook (Chapter 1) Analysis
(AT\_Ch1).''

\subsubsection{Domain 1: Scientific Principles of Substance Use,
Co-Occurring Disorders, and
Recovery}\label{domain-1-scientific-principles-of-substance-use-co-occurring-disorders-and-recovery}

This domain covers the foundational knowledge required to understand
SUDs, related conditions, and the principles of recovery.

\begin{enumerate}
\def\labelenumi{\arabic{enumi}.}
\tightlist
\item
  \textbf{KSAO Name:} Knowledge of SUD Terminology, Progression, and
  Behavioral Patterns

  \begin{itemize}
  \tightlist
  \item
    \textbf{Description:} Comprehensive understanding and correct use of
    current, accurate, and non-stigmatizing terminology in the SUD field
    (e.g., ``SUD,'' ``problematic substance use,'' person-first
    language). Understanding that substance use exists on a continuum
    and the typical stages, common behaviors, patterns, and progressive
    cycles involved in the development of an SUD (e.g., impulsivity,
    reinforcement, tolerance, dependence, craving, withdrawal,
    compulsivity).
  \item
    \textbf{Classification:} Knowledge
  \item
    \textbf{Specificity:} Specialized
  \item
    \textbf{O*NET:} 21-1011.00, 21-1018.00
  \item
    \textbf{Stability/Malleability:} Developable (terminology evolves;
    concepts are learned)
  \item
    \textbf{Explicit/Tacit:} Explicit
  \item
    \textbf{Prerequisites:} Basic literacy. Foundational for all other
    SUD-specific KSAOs.
  \end{itemize}
\item
  \textbf{KSAO Name:} Knowledge of Addiction Models and Theories

  \begin{itemize}
  \tightlist
  \item
    \textbf{Description:} Understanding of various historical and
    contemporary models explaining SUDs, including the disease model
    (SUD as a chronic brain disorder impacting reward, stress,
    self-control), moral model (and its critiques), and
    blended/integrated models that incorporate biological,
    psychological, social factors, and client agency.
  \item
    \textbf{Classification:} Knowledge
  \item
    \textbf{Specificity:} Specialized
  \item
    \textbf{O*NET:} 21-1011.00, 21-1018.00
  \item
    \textbf{Stability/Malleability:} Developable
  \item
    \textbf{Explicit/Tacit:} Explicit
  \item
    \textbf{Prerequisites:} KSAO 1 (Knowledge of SUD Terminology).
    Foundational for conceptualizing treatment.
  \end{itemize}
\item
  \textbf{KSAO Name:} Knowledge of the Neurobiology of SUDs

  \begin{itemize}
  \tightlist
  \item
    \textbf{Description:} In-depth understanding of how addictive
    substances affect brain structure and function. Includes basic
    neuroanatomy/physiology (neurons, neurotransmitters, synapses,
    receptors); key brain regions (basal ganglia, extended amygdala,
    prefrontal cortex) and their roles in SUDs; role of key
    neurotransmitters (e.g., dopamine); mechanisms of drug action;
    neuroadaptations from chronic use; neurobiological basis of
    tolerance, withdrawal, craving, and compulsion; and advanced
    neurobiological models of addiction stages (Binge/Intoxication,
    Withdrawal/Negative Affect, Preoccupation/Anticipation).
  \item
    \textbf{Classification:} Knowledge
  \item
    \textbf{Specificity:} Specialized
  \item
    \textbf{O*NET:} 21-1011.00, 21-1018.00
  \item
    \textbf{Stability/Malleability:} Developable (complex and evolving)
  \item
    \textbf{Explicit/Tacit:} Explicit
  \item
    \textbf{Prerequisites:} KSAO 1, basic biology/chemistry.
    Foundational for understanding SUD treatment and client experience.
  \end{itemize}
\item
  \textbf{KSAO Name:} Knowledge of SUD Etiology, Risk, and Protective
  Factors

  \begin{itemize}
  \tightlist
  \item
    \textbf{Description:} Understanding of the multifactorial etiology
    of SUDs, including genetic predispositions, gender-related
    differences, environmental influences (home, family, peers, school),
    developmental factors (e.g., age of first use), method of substance
    administration, impact of trauma (e.g., ACEs, violence, abuse) and
    its neurobiological effects, and co-occurring mental health
    conditions as risk factors. Understanding of protective factors that
    mitigate risk.
  \item
    \textbf{Classification:} Knowledge
  \item
    \textbf{Specificity:} Specialized
  \item
    \textbf{O*NET:} 21-1011.00, 21-1018.00
  \item
    \textbf{Stability/Malleability:} Developable
  \item
    \textbf{Explicit/Tacit:} Explicit
  \item
    \textbf{Prerequisites:} KSAO 2 (Knowledge of Addiction Models).
    Essential for assessment, prevention, and trauma-informed care.
  \end{itemize}
\item
  \textbf{KSAO Name:} Knowledge of Pharmacology, Drug Classifications,
  and Characteristics of Common Substances of Abuse

  \begin{itemize}
  \tightlist
  \item
    \textbf{Description:} Familiarity with major categories of
    psychoactive substances (e.g., opioids, CNS depressants, stimulants,
    cannabis, hallucinogens, inhalants, dissociatives, nicotine, alcohol
    \textless80\textgreater\textless93\textgreater{} as detailed in
    Appendix B of textbook). Includes commercial/street names, forms,
    methods of administration, intoxication effects, short/long-term
    health effects, withdrawal syndromes, potential for
    medication-assisted treatment (MAT), and understanding of the
    Controlled Substances Act (CSA) and drug schedules.
  \item
    \textbf{Classification:} Knowledge
  \item
    \textbf{Specificity:} Specialized
  \item
    \textbf{O*NET:} 21-1011.00, 21-1018.00
  \item
    \textbf{Stability/Malleability:} Developable (requires ongoing
    learning)
  \item
    \textbf{Explicit/Tacit:} Explicit
  \item
    \textbf{Prerequisites:} Basic pharmacology understanding. Critical
    for assessment, crisis intervention, and client education.
  \end{itemize}
\item
  \textbf{KSAO Name:} Knowledge of Co-occurring Mental and Physical
  Health Conditions

  \begin{itemize}
  \tightlist
  \item
    \textbf{Description:} Understanding of signs, symptoms, and basic
    treatment approaches for mental health conditions frequently
    co-occurring with SUDs (e.g., anxiety, mood disorders, personality
    disorders, PTSD, ADHD
    \textless80\textgreater\textless93\textgreater{} as detailed in
    Appendix C of textbook). Awareness of common medical conditions
    resulting from or exacerbated by substance use (e.g., cirrhosis,
    respiratory deficits, cardiovascular issues, infectious diseases)
    and the implications for integrated assessment and treatment.
  \item
    \textbf{Classification:} Knowledge
  \item
    \textbf{Specificity:} Specialized
  \item
    \textbf{O*NET:} 21-1011.00, 21-1018.00
  \item
    \textbf{Stability/Malleability:} Developable
  \item
    \textbf{Explicit/Tacit:} Explicit
  \item
    \textbf{Prerequisites:} Basic psychology and health literacy.
    Foundational for holistic and integrated care.
  \end{itemize}
\item
  \textbf{KSAO Name:} Knowledge of Recovery Principles, Models, and
  Pathways

  \begin{itemize}
  \tightlist
  \item
    \textbf{Description:} Understanding of contemporary recovery
    concepts, including SAMHSA's definition and four dimensions (Health,
    Home, Purpose, Community), the ten guiding principles of recovery
    (hope, person-centered, many pathways, holistic, peer support,
    etc.). Awareness of diverse approaches to recovery including
    evidence-based treatments (e.g., MAT), mutual-help groups, holistic
    health, faith-based approaches, and harm reduction.
  \item
    \textbf{Classification:} Knowledge
  \item
    \textbf{Specificity:} Specialized
  \item
    \textbf{O*NET:} 21-1011.00, 21-1018.00
  \item
    \textbf{Stability/Malleability:} Developable
  \item
    \textbf{Explicit/Tacit:} Explicit
  \item
    \textbf{Prerequisites:} Understanding of SUDs as chronic conditions.
    Fosters a recovery-oriented mindset.
  \end{itemize}
\item
  \textbf{KSAO Name:} Knowledge of NIDA Principles of Effective SUD
  Treatment

  \begin{itemize}
  \tightlist
  \item
    \textbf{Description:} Understanding of research-based principles for
    effective drug addiction treatment for adults and adolescents as
    outlined by NIDA (e.g., treatment availability, individualized care,
    duration, behavioral therapies, medications, co-occurring disorders,
    monitoring \textless80\textgreater\textless93\textgreater{} as in
    Appendix E of textbook).
  \item
    \textbf{Classification:} Knowledge
  \item
    \textbf{Specificity:} Specialized
  \item
    \textbf{O*NET:} 21-1011.00, 21-1018.00
  \item
    \textbf{Stability/Malleability:} Developable
  \item
    \textbf{Explicit/Tacit:} Explicit
  \item
    \textbf{Prerequisites:} Foundational for evidence-based treatment
    planning.
  \end{itemize}
\item
  \textbf{KSAO Name:} Knowledge of the Transtheoretical Model (Stages of
  Change - SOC)

  \begin{itemize}
  \tightlist
  \item
    \textbf{Description:} Understanding the principles of the SOC model
    (Prochaska \& DiClemente), including the five stages
    (Precontemplation, Contemplation, Preparation, Action, Maintenance),
    the cyclical nature of change, processes of change, and
    interpretation of recurrence/relapse as a learning opportunity.
    Includes understanding of natural recovery.
  \item
    \textbf{Classification:} Knowledge
  \item
    \textbf{Specificity:} General (behavioral health) but applied
    specifically in SUD.
  \item
    \textbf{O*NET:} 21-1011.00, 21-1018.00
  \item
    \textbf{Stability/Malleability:} Developable
  \item
    \textbf{Explicit/Tacit:} Explicit
  \item
    \textbf{Prerequisites:} Basic understanding of behavior change.
    Essential for tailoring interventions.
  \end{itemize}
\item
  \textbf{KSAO Name:} Knowledge of Trauma-Informed Care Principles

  \begin{itemize}
  \tightlist
  \item
    \textbf{Description:} Comprehensive understanding of trauma
    (SAMHSA's ``Three Es'': Events, Experience, Effects), types of
    trauma (acute, chronic, complex, ACEs), the strong connection
    between trauma and SUDs, and the core principles of trauma-informed
    care (safety, trustworthiness, peer support, collaboration,
    empowerment, choice, cultural sensitivity) to avoid
    re-traumatization and support healing.
  \item
    \textbf{Classification:} Knowledge
  \item
    \textbf{Specificity:} Specialized (increasingly general in
    behavioral health)
  \item
    \textbf{O*NET:} 21-1011.00, 21-1018.00
  \item
    \textbf{Stability/Malleability:} Developable
  \item
    \textbf{Explicit/Tacit:} Explicit
  \item
    \textbf{Prerequisites:} KSAO 4 (Risk Factors, including trauma).
    Foundational for Skill in Trauma-Informed Practices.
  \end{itemize}
\item
  \textbf{KSAO Name:} Knowledge of Stigma and Discrimination in SUD

  \begin{itemize}
  \tightlist
  \item
    \textbf{Description:} Understanding the nature, pervasiveness, and
    detrimental effects of prejudice, public stigma, self-stigma, and
    discrimination faced by individuals with SUDs, including its impact
    on treatment seeking, retention, outcomes, providers, research, and
    policy.
  \item
    \textbf{Classification:} Knowledge
  \item
    \textbf{Specificity:} Specialized
  \item
    \textbf{O*NET:} 21-1011.00, 21-1018.00
  \item
    \textbf{Stability/Malleability:} Developable (awareness can be
    cultivated)
  \item
    \textbf{Explicit/Tacit:} Both (explicitly taught, tacitly understood
    through client interactions)
  \item
    \textbf{Prerequisites:} General understanding of social justice.
    Foundational for ethical and empathetic practice.
  \end{itemize}
\item
  \textbf{KSAO Name:} Knowledge of Social Determinants of Health (SDOH)
  and SUD

  \begin{itemize}
  \tightlist
  \item
    \textbf{Description:} Understanding the five domains of SDOH
    (Healthcare access/quality, Education access/quality,
    Social/community context, Economic stability, Neighborhood/built
    environment) and their complex, interdependent relationship with
    problematic substance use, health disparities, health outcomes, and
    recovery.
  \item
    \textbf{Classification:} Knowledge
  \item
    \textbf{Specificity:} Specialized
  \item
    \textbf{O*NET:} 21-1011.00, 21-1018.00
  \item
    \textbf{Stability/Malleability:} Developable
  \item
    \textbf{Explicit/Tacit:} Explicit
  \item
    \textbf{Prerequisites:} Understanding of public health and social
    systems. Important for holistic assessment and advocacy.
  \end{itemize}
\item
  \textbf{KSAO Name:} Knowledge of SUD Treatment Systems, Levels of
  Care, and Modalities

  \begin{itemize}
  \tightlist
  \item
    \textbf{Description:} Comprehensive understanding of the SUD
    treatment service system, its evolution, the continuum of care
    (detoxification, outpatient (standard, IOP, PHP), residential, MAT),
    and criteria for client suitability. Includes knowledge of various
    modalities like individual, group, family counseling, and
    psychoeducation.
  \item
    \textbf{Classification:} Knowledge
  \item
    \textbf{Specificity:} Specialized
  \item
    \textbf{O*NET:} 21-1011.00, 21-1018.00
  \item
    \textbf{Stability/Malleability:} Developable
  \item
    \textbf{Explicit/Tacit:} Explicit
  \item
    \textbf{Prerequisites:} KSAO 1 (Terminology). Essential for
    assessment of LOC and referral.
  \end{itemize}
\item
  \textbf{KSAO Name:} Knowledge of Harm Reduction Principles and
  Strategies

  \begin{itemize}
  \tightlist
  \item
    \textbf{Description:} In-depth understanding of the harm reduction
    philosophy (evidence-based, person-centered, non-coercive, focused
    on reducing negative impacts without requiring abstinence), its core
    pillars and principles. Knowledge of specific strategies like
    overdose education, naloxone distribution, fentanyl/xylazine test
    strips, safer use practices, and syringe services programs (SSPs).
  \item
    \textbf{Classification:} Knowledge
  \item
    \textbf{Specificity:} Specialized
  \item
    \textbf{O*NET:} 21-1011.00, 21-1018.00
  \item
    \textbf{Stability/Malleability:} Developable
  \item
    \textbf{Explicit/Tacit:} Explicit
  \item
    \textbf{Prerequisites:} KSAO 11 (Stigma), openness to
    non-abstinence-only approaches.
  \end{itemize}
\item
  \textbf{KSAO Name:} Knowledge of Mutual-Help Groups

  \begin{itemize}
  \tightlist
  \item
    \textbf{Description:} Understanding the nature, function,
    principles, and variety of mutual-help groups (e.g., 12-Step
    programs like AA/NA, SMART Recovery, Women for Sobriety), including
    their member-led structure, focus on shared experience, and role in
    the recovery process alongside professional treatment.
  \item
    \textbf{Classification:} Knowledge
  \item
    \textbf{Specificity:} Specialized
  \item
    \textbf{O*NET:} 21-1011.00, 21-1018.00
  \item
    \textbf{Stability/Malleability:} Developable
  \item
    \textbf{Explicit/Tacit:} Explicit
  \item
    \textbf{Prerequisites:} Awareness of support systems. Supports
    referral and comprehensive recovery planning.
  \end{itemize}
\end{enumerate}

\subsubsection{Domain 2: Evidence-Based Screening, Assessment, and
Diagnosis}\label{domain-2-evidence-based-screening-assessment-and-diagnosis}

This domain focuses on the competencies needed to evaluate clients'
needs and formulate diagnoses.

\begin{enumerate}
\def\labelenumi{\arabic{enumi}.}
\setcounter{enumi}{15}
\tightlist
\item
  \textbf{KSAO Name:} Knowledge of Clinical Interviewing Techniques

  \begin{itemize}
  \tightlist
  \item
    \textbf{Description:} Understanding of established clinical
    interviewing techniques, including Motivational Interviewing (MI),
    active listening, probing, open/closed questioning, reflection,
    summarizing, and stages of the clinical interview.
  \item
    \textbf{Classification:} Knowledge
  \item
    \textbf{Specificity:} General (core counseling), Specialized (MI in
    SUD)
  \item
    \textbf{O*NET:} 21-1011.00, 21-1018.00, 21-1015.00
  \item
    \textbf{Stability/Malleability:} Developable
  \item
    \textbf{Explicit/Tacit:} Explicit (principles taught)
  \item
    \textbf{Prerequisites:} Basic communication skills. Foundational for
    Skill in Clinical Interviewing.
  \end{itemize}
\item
  \textbf{KSAO Name:} Skill in Clinical Interviewing and Information
  Gathering

  \begin{itemize}
  \tightlist
  \item
    \textbf{Description:} Proficient application of various interviewing
    techniques to gather comprehensive client information (substance use
    history, mental/physical health, social context, strengths, etc.),
    build rapport, explore ambivalence, assess risk, and facilitate
    client engagement in the screening, assessment, and diagnostic
    process.
  \item
    \textbf{Classification:} Skill
  \item
    \textbf{Specificity:} Specialized
  \item
    \textbf{O*NET:} 21-1011.00, 21-1018.00, 21-1015.00
  \item
    \textbf{Stability/Malleability:} Developable (through practice and
    feedback)
  \item
    \textbf{Explicit/Tacit:} Both (techniques are explicit, empathetic
    application is more tacit)
  \item
    \textbf{Prerequisites:} KSAO 16, KSAO 46 (Skill in Therapeutic
    Communication), O6 (Empathy).
  \end{itemize}
\item
  \textbf{KSAO Name:} Knowledge of SUD Screening and Assessment
  Instruments

  \begin{itemize}
  \tightlist
  \item
    \textbf{Description:} Familiarity with established, valid, and
    reliable evidence-based screening and assessment tools used in SUD
    settings (e.g., AUDIT, DAST, ASSIST, ASI, ACE, SASSI). Includes
    understanding their purpose, administration, scoring, psychometric
    properties, and cultural considerations.
  \item
    \textbf{Classification:} Knowledge
  \item
    \textbf{Specificity:} Specialized
  \item
    \textbf{O*NET:} 21-1011.00, 21-1018.00
  \item
    \textbf{Stability/Malleability:} Developable
  \item
    \textbf{Explicit/Tacit:} Explicit
  \item
    \textbf{Prerequisites:} Foundational for Skill in Administering and
    Interpreting Instruments.
  \end{itemize}
\item
  \textbf{KSAO Name:} Skill in Administering and Interpreting SUD
  Screening/Assessment Instruments

  \begin{itemize}
  \tightlist
  \item
    \textbf{Description:} Proficiency in selecting, administering,
    scoring, and interpreting appropriate screening and assessment
    instruments to identify potential SUDs, assess severity, identify
    co-occurring issues, and inform diagnostic formulation and treatment
    planning, considering cultural factors.
  \item
    \textbf{Classification:} Skill
  \item
    \textbf{Specificity:} Specialized
  \item
    \textbf{O*NET:} 21-1011.00, 21-1018.00
  \item
    \textbf{Stability/Malleability:} Developable
  \item
    \textbf{Explicit/Tacit:} Explicit (administration/scoring), Tacit
    (clinical interpretation in context)
  \item
    \textbf{Prerequisites:} KSAO 18.
  \end{itemize}
\item
  \textbf{KSAO Name:} Knowledge of Drug and Alcohol Testing Methods

  \begin{itemize}
  \tightlist
  \item
    \textbf{Description:} Understanding of various methods for drug and
    alcohol testing (e.g., urine, breath, blood, saliva, hair),
    including their detection windows, accuracy, reliability,
    limitations, and potential for tampering or false
    positives/negatives.
  \item
    \textbf{Classification:} Knowledge
  \item
    \textbf{Specificity:} Specialized
  \item
    \textbf{O*NET:} 21-1011.00, 21-1018.00
  \item
    \textbf{Stability/Malleability:} Developable
  \item
    \textbf{Explicit/Tacit:} Explicit
  \item
    \textbf{Prerequisites:} Basic understanding of biology/chemistry.
  \end{itemize}
\item
  \textbf{KSAO Name:} Skill in Interpreting and Utilizing Drug and
  Alcohol Test Results

  \begin{itemize}
  \tightlist
  \item
    \textbf{Description:} Ability to accurately interpret results from
    drug and alcohol tests in the context of other client information
    (self-report, clinical presentation), understand their implications
    for treatment, and discuss results with clients therapeutically and
    ethically.
  \item
    \textbf{Classification:} Skill
  \item
    \textbf{Specificity:} Specialized
  \item
    \textbf{O*NET:} 21-1011.00, 21-1018.00
  \item
    \textbf{Stability/Malleability:} Developable
  \item
    \textbf{Explicit/Tacit:} Explicit (understanding results), Tacit
    (contextual interpretation, client communication)
  \item
    \textbf{Prerequisites:} KSAO 20.
  \end{itemize}
\item
  \textbf{KSAO Name:} Knowledge of Diagnostic Criteria for SUDs and
  Co-occurring Disorders

  \begin{itemize}
  \tightlist
  \item
    \textbf{Description:} Thorough understanding of the Diagnostic and
    Statistical Manual of Mental Disorders (DSM; currently DSM-5-TR) or
    other relevant diagnostic criteria (e.g., ICD) for substance use
    disorders and common co-occurring mental health disorders. This
    includes criteria, specifiers for severity, course, and remission
    (as supported by Appendix D of textbook).
  \item
    \textbf{Classification:} Knowledge
  \item
    \textbf{Specificity:} Specialized
  \item
    \textbf{O*NET:} 21-1011.00, 21-1018.00
  \item
    \textbf{Stability/Malleability:} Developable (criteria are updated)
  \item
    \textbf{Explicit/Tacit:} Explicit
  \item
    \textbf{Prerequisites:} KSAO 1, KSAO 6.
  \end{itemize}
\item
  \textbf{KSAO Name:} Skill in SUD Assessment, Diagnosis, and Case
  Conceptualization

  \begin{itemize}
  \tightlist
  \item
    \textbf{Description:} Ability to synthesize comprehensive client
    information (biopsychosocial history, substance use patterns,
    screening/assessment results, collateral information, cultural
    context, strengths) and apply diagnostic criteria (DSM/ICD) to
    formulate an accurate diagnosis and a holistic case
    conceptualization that explains the client's presenting problems and
    informs treatment.
  \item
    \textbf{Classification:} Skill
  \item
    \textbf{Specificity:} Specialized
  \item
    \textbf{O*NET:} 21-1011.00, 21-1018.00
  \item
    \textbf{Stability/Malleability:} Developable (requires supervised
    experience)
  \item
    \textbf{Explicit/Tacit:} Both (applying explicit criteria, tacit
    integration and judgment)
  \item
    \textbf{Prerequisites:} KSAO 17, KSAO 19, KSAO 22, A1 (Ability to
    Synthesize Complex Information).
  \end{itemize}
\item
  \textbf{KSAO Name:} Knowledge of Level of Care Placement Criteria
  (e.g., ASAM)

  \begin{itemize}
  \tightlist
  \item
    \textbf{Description:} Understanding of established multidimensional
    criteria (e.g., ASAM Criteria) for determining the appropriate
    intensity and level of care for individuals with SUDs based on
    assessed needs, severity, risks, and resources.
  \item
    \textbf{Classification:} Knowledge
  \item
    \textbf{Specificity:} Specialized
  \item
    \textbf{O*NET:} 21-1011.00, 21-1018.00
  \item
    \textbf{Stability/Malleability:} Developable
  \item
    \textbf{Explicit/Tacit:} Explicit
  \item
    \textbf{Prerequisites:} KSAO 13.
  \end{itemize}
\item
  \textbf{KSAO Name:} Skill in Determining Level of Care and Service
  Coordination

  \begin{itemize}
  \tightlist
  \item
    \textbf{Description:} Ability to apply placement criteria and
    clinical judgment to match clients to the appropriate level of care
    that meets their immediate and ongoing needs, and to coordinate
    services effectively within the system of care.
  \item
    \textbf{Classification:} Skill
  \item
    \textbf{Specificity:} Specialized
  \item
    \textbf{O*NET:} 21-1011.00, 21-1018.00
  \item
    \textbf{Stability/Malleability:} Developable
  \item
    \textbf{Explicit/Tacit:} Both (criteria application is explicit,
    judgment and coordination are tacit)
  \item
    \textbf{Prerequisites:} KSAO 23, KSAO 24, KSAO 26 (Skill in Clinical
    Decision-Making).
  \end{itemize}
\item
  \textbf{KSAO Name:} Skill in Clinical Decision-Making (Assessment \&
  Initial Planning)

  \begin{itemize}
  \tightlist
  \item
    \textbf{Description:} Ability to analyze assessment data, identify
    client strengths, needs, and stage of change, prioritize problems,
    and determine an appropriate initial course of action, including
    immediate safety needs and preliminary treatment goals.
  \item
    \textbf{Classification:} Skill
  \item
    \textbf{Specificity:} Specialized
  \item
    \textbf{O*NET:} 21-1011.00, 21-1018.00
  \item
    \textbf{Stability/Malleability:} Developable (through experience and
    supervision)
  \item
    \textbf{Explicit/Tacit:} Both (relies on explicit knowledge and
    tacit experience/judgment)
  \item
    \textbf{Prerequisites:} All preceding KSAOs in Domain 1 \& 2, A2
    (Ability for Critical Thinking).
  \end{itemize}
\end{enumerate}

\subsubsection{Domain 3: Evidence-Based Treatment, Counseling, and
Referral}\label{domain-3-evidence-based-treatment-counseling-and-referral}

This domain covers the competencies related to providing direct client
services and facilitating connections to other resources.

\begin{enumerate}
\def\labelenumi{\arabic{enumi}.}
\setcounter{enumi}{26}
\tightlist
\item
  \textbf{KSAO Name:} Knowledge of Evidence-Based Practices (EBPs) in
  SUD Treatment

  \begin{itemize}
  \tightlist
  \item
    \textbf{Description:} Understanding of specific therapeutic
    modalities, interventions, and relational stances (e.g., MI, CBT,
    Contingency Management, MAT, community reinforcement approach,
    family therapy models) that have been scientifically validated as
    effective for treating SUDs and co-occurring disorders. Knowledge of
    where to find EBP resources (e.g., NIDA, SAMHSA, as in Appendix F of
    textbook).
  \item
    \textbf{Classification:} Knowledge
  \item
    \textbf{Specificity:} Specialized
  \item
    \textbf{O*NET:} 21-1011.00, 21-1018.00
  \item
    \textbf{Stability/Malleability:} Developable (requires ongoing
    learning)
  \item
    \textbf{Explicit/Tacit:} Explicit
  \item
    \textbf{Prerequisites:} Foundational scientific literacy, Domain 1
    KSAOs.
  \end{itemize}
\item
  \textbf{KSAO Name:} Skill in Applying Evidence-Based Counseling
  Practices

  \begin{itemize}
  \tightlist
  \item
    \textbf{Description:} Proficiency in selecting and implementing
    evidence-based counseling techniques and interventions appropriate
    to client needs, preferences, cultural background, stage of change,
    and treatment goals, with fidelity to the model where applicable,
    and adapting as necessary.
  \item
    \textbf{Classification:} Skill
  \item
    \textbf{Specificity:} Specialized
  \item
    \textbf{O*NET:} 21-1011.00, 21-1018.00
  \item
    \textbf{Stability/Malleability:} Developable (through training and
    supervised practice)
  \item
    \textbf{Explicit/Tacit:} Both (models are explicit, artful
    application is tacit)
  \item
    \textbf{Prerequisites:} KSAO 27, relevant KSAOs for specific EBPs
    (e.g., KSAO 16 for MI).
  \end{itemize}
\item
  \textbf{KSAO Name:} Knowledge of Treatment Planning Principles and
  Components

  \begin{itemize}
  \tightlist
  \item
    \textbf{Description:} Understanding of best practices in developing
    collaborative, client-centered, strengths-based, and culturally
    responsive treatment plans. Includes identifying problems, setting
    measurable goals and objectives (SMART goals), selecting appropriate
    strategies/interventions, and planning for review and modification.
  \item
    \textbf{Classification:} Knowledge
  \item
    \textbf{Specificity:} Specialized
  \item
    \textbf{O*NET:} 21-1011.00, 21-1018.00
  \item
    \textbf{Stability/Malleability:} Developable
  \item
    \textbf{Explicit/Tacit:} Explicit
  \item
    \textbf{Prerequisites:} KSAOs from Domains 1 \& 2. Foundational for
    Skill in Treatment Planning.
  \end{itemize}
\item
  \textbf{KSAO Name:} Skill in Collaborative Treatment Planning and Goal
  Setting

  \begin{itemize}
  \tightlist
  \item
    \textbf{Description:} Proficiency in collaboratively creating,
    implementing, monitoring, and modifying individualized treatment
    plans that address client needs, strengths, goals, preferences, and
    stage of change, incorporating evidence-based strategies, recovery
    principles, and cultural considerations.
  \item
    \textbf{Classification:} Skill
  \item
    \textbf{Specificity:} Specialized
  \item
    \textbf{O*NET:} 21-1011.00, 21-1018.00
  \item
    \textbf{Stability/Malleability:} Developable
  \item
    \textbf{Explicit/Tacit:} Both (structure is explicit, collaboration
    and individualization are tacit)
  \item
    \textbf{Prerequisites:} KSAO 29, KSAO 23, KSAO 26, KSAO 46
    (Therapeutic Communication).
  \end{itemize}
\item
  \textbf{KSAO Name:} Knowledge of Relapse Prevention Strategies

  \begin{itemize}
  \tightlist
  \item
    \textbf{Description:} Understanding of theories, models (e.g.,
    Marlatt's cognitive-behavioral model), and techniques for preventing
    relapse, including identifying high-risk situations, triggers,
    warning signs, developing coping strategies, managing cravings, and
    addressing the abstinence violation effect.
  \item
    \textbf{Classification:} Knowledge
  \item
    \textbf{Specificity:} Specialized
  \item
    \textbf{O*NET:} 21-1011.00, 21-1018.00
  \item
    \textbf{Stability/Malleability:} Developable
  \item
    \textbf{Explicit/Tacit:} Explicit
  \item
    \textbf{Prerequisites:} KSAO 1, KSAO 3.
  \end{itemize}
\item
  \textbf{KSAO Name:} Knowledge of Coping Skills for SUD Recovery

  \begin{itemize}
  \tightlist
  \item
    \textbf{Description:} Understanding of a variety of cognitive,
    behavioral, emotional regulation, and interpersonal skills that
    support recovery from SUDs (e.g., stress management,
    problem-solving, communication skills, mindfulness, distress
    tolerance, anger management).
  \item
    \textbf{Classification:} Knowledge
  \item
    \textbf{Specificity:} Specialized
  \item
    \textbf{O*NET:} 21-1011.00, 21-1018.00
  \item
    \textbf{Stability/Malleability:} Developable
  \item
    \textbf{Explicit/Tacit:} Explicit
  \item
    \textbf{Prerequisites:} Basic understanding of psychology.
  \end{itemize}
\item
  \textbf{KSAO Name:} Skill in Teaching Coping and Relapse Prevention
  Skills

  \begin{itemize}
  \tightlist
  \item
    \textbf{Description:} Ability to effectively teach, model,
    role-play, and reinforce coping skills and relapse prevention
    strategies with individuals and groups, tailoring approaches to
    client learning styles and needs.
  \item
    \textbf{Classification:} Skill
  \item
    \textbf{Specificity:} Specialized
  \item
    \textbf{O*NET:} 21-1011.00, 21-1018.00
  \item
    \textbf{Stability/Malleability:} Developable
  \item
    \textbf{Explicit/Tacit:} Both (content is explicit, teaching methods
    can be tacit)
  \item
    \textbf{Prerequisites:} KSAO 31, KSAO 32, KSAO 48 (Skill in
    Psychoeducation).
  \end{itemize}
\item
  \textbf{KSAO Name:} Knowledge of Group Counseling Theories, Dynamics,
  and Techniques

  \begin{itemize}
  \tightlist
  \item
    \textbf{Description:} Understanding of theories, models, stages of
    group development, therapeutic factors, group dynamics (cohesion,
    conflict, roles), and techniques specific to facilitating SUD group
    counseling, including structured curricula and process-oriented
    groups.
  \item
    \textbf{Classification:} Knowledge
  \item
    \textbf{Specificity:} Specialized
  \item
    \textbf{O*NET:} 21-1011.00, 21-1018.00
  \item
    \textbf{Stability/Malleability:} Developable
  \item
    \textbf{Explicit/Tacit:} Explicit
  \item
    \textbf{Prerequisites:} Foundational counseling knowledge. Essential
    for Skill in Group Facilitation.
  \end{itemize}
\item
  \textbf{KSAO Name:} Skill in Group Facilitation and Management

  \begin{itemize}
  \tightlist
  \item
    \textbf{Description:} Proficiency in leading and managing SUD group
    counseling sessions, fostering a safe/supportive/respectful
    environment, managing group dynamics, promoting cohesiveness,
    addressing challenging behaviors, and guiding the group towards
    therapeutic goals.
  \item
    \textbf{Classification:} Skill
  \item
    \textbf{Specificity:} Specialized
  \item
    \textbf{O*NET:} 21-1011.00, 21-1018.00
  \item
    \textbf{Stability/Malleability:} Developable (through training and
    supervised experience)
  \item
    \textbf{Explicit/Tacit:} Both (techniques are explicit, managing
    dynamics in real-time is tacit)
  \item
    \textbf{Prerequisites:} KSAO 34, KSAO 46 (Therapeutic
    Communication).
  \end{itemize}
\item
  \textbf{KSAO Name:} Knowledge of Strategies for Addressing Client
  Ambivalence and Resistance (Motivational Interviewing)

  \begin{itemize}
  \tightlist
  \item
    \textbf{Description:} Understanding of theoretical frameworks (e.g.,
    Stages of Change, Motivational Interviewing) and specific techniques
    for working effectively with clients who are ambivalent or resistant
    to change, including recognizing change talk and sustain talk.
  \item
    \textbf{Classification:} Knowledge
  \item
    \textbf{Specificity:} Specialized
  \item
    \textbf{O*NET:} 21-1011.00, 21-1018.00
  \item
    \textbf{Stability/Malleability:} Developable
  \item
    \textbf{Explicit/Tacit:} Explicit
  \item
    \textbf{Prerequisites:} KSAO 9, KSAO 16.
  \end{itemize}
\item
  \textbf{KSAO Name:} Skill in Addressing Client Ambivalence and
  Resistance (Motivational Interviewing)

  \begin{itemize}
  \tightlist
  \item
    \textbf{Description:} Proficiency in applying MI principles and
    techniques (e.g., OARS, rolling with resistance, developing
    discrepancy, supporting self-efficacy) to explore and resolve client
    ambivalence and gently work through resistance to change.
  \item
    \textbf{Classification:} Skill
  \item
    \textbf{Specificity:} Specialized
  \item
    \textbf{O*NET:} 21-1011.00, 21-1018.00
  \item
    \textbf{Stability/Malleability:} Developable
  \item
    \textbf{Explicit/Tacit:} Both (techniques are explicit, fluid
    application is tacit)
  \item
    \textbf{Prerequisites:} KSAO 36, KSAO 17.
  \end{itemize}
\item
  \textbf{KSAO Name:} Knowledge of Crisis Intervention and De-escalation
  Principles

  \begin{itemize}
  \tightlist
  \item
    \textbf{Description:} Understanding of the signs of escalating
    psychological or behavioral crisis situations (including those
    related to acute intoxication, withdrawal, or psychiatric distress),
    principles of crisis intervention, risk assessment
    (suicide/homicide), and verbal/non-verbal de-escalation techniques.
  \item
    \textbf{Classification:} Knowledge
  \item
    \textbf{Specificity:} Specialized (high stakes in SUD settings)
  \item
    \textbf{O*NET:} 21-1011.00, 21-1018.00
  \item
    \textbf{Stability/Malleability:} Developable
  \item
    \textbf{Explicit/Tacit:} Explicit
  \item
    \textbf{Prerequisites:} KSAO 5, KSAO 6.
  \end{itemize}
\item
  \textbf{KSAO Name:} Skill in Crisis Intervention and De-escalation

  \begin{itemize}
  \tightlist
  \item
    \textbf{Description:} Ability to recognize, assess, and respond
    effectively to emergency or crisis situations, including
    implementing de-escalation techniques, ensuring safety for
    clients/staff, and making appropriate referrals for higher-level
    care or emergency services.
  \item
    \textbf{Classification:} Skill
  \item
    \textbf{Specificity:} Specialized
  \item
    \textbf{O*NET:} 21-1011.00, 21-1018.00
  \item
    \textbf{Stability/Malleability:} Developable (through training and
    simulation)
  \item
    \textbf{Explicit/Tacit:} Both (protocols are explicit, real-time
    judgment is tacit)
  \item
    \textbf{Prerequisites:} KSAO 38, O1 (Self-Awareness/Composure).
  \end{itemize}
\item
  \textbf{KSAO Name:} Knowledge of Discharge, Continuing Care, and
  Termination Processes

  \begin{itemize}
  \tightlist
  \item
    \textbf{Description:} Understanding of components of comprehensive
    discharge planning (continuing care recommendations, relapse
    prevention, referrals for ongoing support, coordination with
    community resources) and ethical/clinical considerations for
    terminating the counseling relationship.
  \item
    \textbf{Classification:} Knowledge
  \item
    \textbf{Specificity:} Specialized
  \item
    \textbf{O*NET:} 21-1011.00, 21-1018.00
  \item
    \textbf{Stability/Malleability:} Developable
  \item
    \textbf{Explicit/Tacit:} Explicit
  \item
    \textbf{Prerequisites:} KSAO 13, KSAO 42 (Knowledge of Community
    Resources).
  \end{itemize}
\item
  \textbf{KSAO Name:} Skill in Discharge, Continuing Care Planning, and
  Termination

  \begin{itemize}
  \tightlist
  \item
    \textbf{Description:} Ability to collaboratively develop
    comprehensive discharge and continuing care plans with clients, make
    appropriate referrals, and ethically terminate the counseling
    relationship, ensuring a smooth transition and support for sustained
    recovery.
  \item
    \textbf{Classification:} Skill
  \item
    \textbf{Specificity:} Specialized
  \item
    \textbf{O*NET:} 21-1011.00, 21-1018.00
  \item
    \textbf{Stability/Malleability:} Developable
  \item
    \textbf{Explicit/Tacit:} Both
  \item
    \textbf{Prerequisites:} KSAO 40, KSAO 43 (Skill in Making
    Referrals).
  \end{itemize}
\item
  \textbf{KSAO Name:} Knowledge of Community Resources and Referral
  Processes

  \begin{itemize}
  \tightlist
  \item
    \textbf{Description:} Understanding of how to identify appropriate
    community resources (e.g., for housing, employment, medical care,
    mental health services, legal aid, mutual-help, harm reduction
    services, primary care, FTCs) and the processes for making effective
    referrals and facilitating client access. This includes knowledge of
    systems integration and collaborative care models.
  \item
    \textbf{Classification:} Knowledge
  \item
    \textbf{Specificity:} Specialized (for SUD-relevant resources and
    systems)
  \item
    \textbf{O*NET:} 21-1011.00, 21-1018.00
  \item
    \textbf{Stability/Malleability:} Developable (requires ongoing
    updating)
  \item
    \textbf{Explicit/Tacit:} Explicit
  \item
    \textbf{Prerequisites:} KSAO 13.
  \end{itemize}
\item
  \textbf{KSAO Name:} Skill in Making Effective Client Referrals and
  Care Coordination

  \begin{itemize}
  \tightlist
  \item
    \textbf{Description:} Ability to identify client needs beyond the
    scope of current services, match clients with appropriate community
    resources, facilitate the referral process effectively, advocate for
    client access, and coordinate care with other providers and systems
    (e.g., primary care, criminal justice, child welfare).
  \item
    \textbf{Classification:} Skill
  \item
    \textbf{Specificity:} Specialized
  \item
    \textbf{O*NET:} 21-1011.00, 21-1018.00
  \item
    \textbf{Stability/Malleability:} Developable
  \item
    \textbf{Explicit/Tacit:} Both (process is explicit, navigation and
    advocacy can be tacit)
  \item
    \textbf{Prerequisites:} KSAO 42, KSAO 44 (Skill in Interprofessional
    Collaboration).
  \end{itemize}
\item
  \textbf{KSAO Name:} Skill in Interprofessional and Interagency
  Collaboration

  \begin{itemize}
  \tightlist
  \item
    \textbf{Description:} Ability to work effectively, respectfully, and
    communicatively with multidisciplinary teams, other healthcare
    professionals (primary care, mental health), and human service
    systems (e.g., criminal justice, child welfare, SMVF services), and
    client support systems (e.g., family) to coordinate care, share
    information appropriately (within confidentiality limits), and
    achieve common treatment goals.
  \item
    \textbf{Classification:} Skill
  \item
    \textbf{Specificity:} General (important in most healthcare)
  \item
    \textbf{O*NET:} 21-1011.00, 21-1018.00, 21-1015.00
  \item
    \textbf{Stability/Malleability:} Developable
  \item
    \textbf{Explicit/Tacit:} Both
  \item
    \textbf{Prerequisites:} KSAO 46 (Therapeutic Communication),
    knowledge of roles of other professionals.
  \end{itemize}
\item
  \textbf{KSAO Name:} Knowledge of Specific Population Needs in SUD
  Treatment

  \begin{itemize}
  \tightlist
  \item
    \textbf{Description:} Understanding the unique challenges,
    strengths, cultural factors, historical trauma, and treatment
    considerations for diverse populations (e.g., based on
    race/ethnicity, LGBTQ+, pregnant individuals, youth, older adults,
    justice-involved, housing insecure, veterans/SMVF).
  \item
    \textbf{Classification:} Knowledge
  \item
    \textbf{Specificity:} Specialized
  \item
    \textbf{O*NET:} 21-1011.00, 21-1018.00
  \item
    \textbf{Stability/Malleability:} Developable (requires ongoing
    learning)
  \item
    \textbf{Explicit/Tacit:} Explicit
  \item
    \textbf{Prerequisites:} KSAO 55 (Knowledge of Multicultural
    Counseling), O2 (Cultural Humility).
  \end{itemize}
\item
  \textbf{KSAO Name:} Skill in Therapeutic Communication and Rapport
  Building

  \begin{itemize}
  \tightlist
  \item
    \textbf{Description:} Proficient use of verbal (e.g., active
    listening, reflecting, questioning, summarizing, providing feedback,
    reframing, clarifying) and non-verbal communication skills to
    establish and maintain a strong therapeutic alliance (rapport)
    characterized by trust, empathy, genuineness, and respect, thereby
    fostering client engagement and positive outcomes. Includes ability
    to instill hope.
  \item
    \textbf{Classification:} Skill
  \item
    \textbf{Specificity:} General (core counseling skill)
  \item
    \textbf{O*NET:} 21-1011.00, 21-1018.00, 21-1015.00
  \item
    \textbf{Stability/Malleability:} Developable
  \item
    \textbf{Explicit/Tacit:} Both (techniques are explicit, empathetic
    connection is tacit)
  \item
    \textbf{Prerequisites:} Basic communication ability, O6 (Empathy).
    Foundational for all counseling interactions.
  \end{itemize}
\item
  \textbf{KSAO Name:} Skill in Applying Person-Centered and
  Trauma-Informed Practices

  \begin{itemize}
  \tightlist
  \item
    \textbf{Description:} Ability to tailor assessment, treatment, and
    support services to the individual needs, preferences, goals, and
    cultural background of the client (``meeting them where they are''),
    respecting their autonomy. Ability to apply principles of
    trauma-informed care in all interactions to create safety, build
    trust, and empower clients.
  \item
    \textbf{Classification:} Skill
  \item
    \textbf{Specificity:} General (to counseling) but highly emphasized
    in SUD
  \item
    \textbf{O*NET:} 21-1011.00, 21-1018.00
  \item
    \textbf{Stability/Malleability:} Developable
  \item
    \textbf{Explicit/Tacit:} Both
  \item
    \textbf{Prerequisites:} KSAO 7, KSAO 10, KSAO 46, O6 (Empathy).
  \end{itemize}
\item
  \textbf{KSAO Name:} Skill in Psychoeducation Delivery

  \begin{itemize}
  \tightlist
  \item
    \textbf{Description:} Ability to effectively provide clients,
    families, and communities with information and education about SUDs,
    neurobiology, risks, treatment options, recovery processes, harm
    reduction, coping skills, and relapse prevention in a clear,
    accessible, culturally sensitive, and engaging manner, individually
    or in groups.
  \item
    \textbf{Classification:} Skill
  \item
    \textbf{Specificity:} Specialized (content is SUD-specific)
  \item
    \textbf{O*NET:} 21-1011.00, 21-1018.00
  \item
    \textbf{Stability/Malleability:} Developable
  \item
    \textbf{Explicit/Tacit:} Both (content is explicit, delivery methods
    can be tacit)
  \item
    \textbf{Prerequisites:} Relevant Domain 1 KSAOs, KSAO 46
    (Therapeutic Communication).
  \end{itemize}
\item
  \textbf{KSAO Name:} Skill in Harm Reduction Application and Education

  \begin{itemize}
  \tightlist
  \item
    \textbf{Description:} Ability to collaboratively discuss and
    implement harm reduction strategies with clients, educate them on
    safer use practices (e.g., naloxone use, fentanyl test strips, safer
    injection), and connect them with harm reduction resources and
    services (e.g., SSPs), respecting client autonomy and goals.
  \item
    \textbf{Classification:} Skill
  \item
    \textbf{Specificity:} Specialized
  \item
    \textbf{O*NET:} 21-1011.00, 21-1018.00
  \item
    \textbf{Stability/Malleability:} Developable
  \item
    \textbf{Explicit/Tacit:} Both
  \item
    \textbf{Prerequisites:} KSAO 14, KSAO 47 (Person-Centered Approach).
  \end{itemize}
\item
  \textbf{KSAO Name:} Skill in Telehealth Service Delivery

  \begin{itemize}
  \tightlist
  \item
    \textbf{Description:} The ability to effectively use chosen
    telehealth technologies to provide counseling services, engage
    clients virtually, adapt clinical skills (assessment,
    individual/group therapy) to a virtual environment, and adhere to
    relevant legal, ethical, and technological standards.
  \item
    \textbf{Classification:} Skill
  \item
    \textbf{Specificity:} Specialized
  \item
    \textbf{O*NET:} 21-1011.00, 21-1018.00
  \item
    \textbf{Stability/Malleability:} Developable
  \item
    \textbf{Explicit/Tacit:} Both (technical aspects are explicit,
    virtual engagement is more tacit)
  \item
    \textbf{Prerequisites:} KSAO 46, Knowledge of Telehealth (KSAO 59),
    A4 (Ability to Engage Clients Virtually), A5 (Technological
    Adaptability).
  \end{itemize}
\end{enumerate}

\subsubsection{Domain 4: Professional, Ethical, and Legal
Responsibilities}\label{domain-4-professional-ethical-and-legal-responsibilities}

This domain covers the competencies related to professional conduct,
adherence to standards, and self-management.

\begin{enumerate}
\def\labelenumi{\arabic{enumi}.}
\setcounter{enumi}{50}
\tightlist
\item
  \textbf{KSAO Name:} Knowledge of Clinical Documentation Standards and
  Practices

  \begin{itemize}
  \tightlist
  \item
    \textbf{Description:} Understanding of best practices, legal
    requirements (e.g., related to 42 CFR Part 2, HIPAA), and ethical
    guidelines for clinical documentation, including progress notes,
    treatment plans, assessment summaries, and discharge summaries, as
    well as record keeping, storage, and sharing protocols.
  \item
    \textbf{Classification:} Knowledge
  \item
    \textbf{Specificity:} Specialized (SUD documentation has unique
    requirements)
  \item
    \textbf{O*NET:} 21-1011.00, 21-1018.00
  \item
    \textbf{Stability/Malleability:} Developable
  \item
    \textbf{Explicit/Tacit:} Explicit
  \item
    \textbf{Prerequisites:} Basic writing skills.
  \end{itemize}
\item
  \textbf{KSAO Name:} Skill in Clinical Documentation

  \begin{itemize}
  \tightlist
  \item
    \textbf{Description:} Proficiency in creating accurate, objective,
    concise, timely, and legally/ethically sound clinical documentation
    that reflects client progress, services provided, treatment
    rationale, and meets regulatory/agency standards.
  \item
    \textbf{Classification:} Skill
  \item
    \textbf{Specificity:} General (core professional skill) but applied
    to SUD context.
  \item
    \textbf{O*NET:} 21-1011.00, 21-1018.00
  \item
    \textbf{Stability/Malleability:} Developable (through training and
    practice)
  \item
    \textbf{Explicit/Tacit:} Explicit (formats, requirements), Tacit
    (nuance of effective note writing)
  \item
    \textbf{Prerequisites:} KSAO 51.
  \end{itemize}
\item
  \textbf{KSAO Name:} Knowledge of Professional Ethics, Boundaries, and
  Scope of Practice

  \begin{itemize}
  \tightlist
  \item
    \textbf{Description:} Comprehensive understanding of ethical codes
    relevant to SUD counseling (e.g., NAADAC, state licensure board
    codes), principles of professional conduct, maintaining appropriate
    therapeutic boundaries, avoiding dual relationships, guidelines for
    self-disclosure, recognizing and managing conflicts of interest, and
    understanding the defined limits of an SUD counselor's scope of
    practice.
  \item
    \textbf{Classification:} Knowledge
  \item
    \textbf{Specificity:} Specialized (for SUD counseling ethics and
    scope)
  \item
    \textbf{O*NET:} 21-1011.00, 21-1018.00
  \item
    \textbf{Stability/Malleability:} Developable
  \item
    \textbf{Explicit/Tacit:} Explicit
  \item
    \textbf{Prerequisites:} Foundational for ethical practice.
  \end{itemize}
\item
  \textbf{KSAO Name:} Skill in Maintaining Professional Boundaries and
  Adhering to Scope of Practice

  \begin{itemize}
  \tightlist
  \item
    \textbf{Description:} Ability to establish and consistently maintain
    appropriate professional boundaries with clients. Ability to
    recognize issues or client needs that fall outside one's scope of
    practice or competence and respond appropriately (e.g.,
    consultation, referral). Skill in identifying and managing conflicts
    of interest ethically.
  \item
    \textbf{Classification:} Skill
  \item
    \textbf{Specificity:} General (core professional skill) but applied
    in SUD context.
  \item
    \textbf{O*NET:} 21-1011.00, 21-1018.00, 21-1015.00
  \item
    \textbf{Stability/Malleability:} Developable (influenced by
    self-awareness and supervision)
  \item
    \textbf{Explicit/Tacit:} Both (principles are explicit, application
    involves judgment and self-regulation)
  \item
    \textbf{Prerequisites:} KSAO 53, O1 (Self-Awareness).
  \end{itemize}
\item
  \textbf{KSAO Name:} Knowledge of Multicultural Counseling Principles
  and Cultural Humility

  \begin{itemize}
  \tightlist
  \item
    \textbf{Description:} Understanding of theories and principles of
    multicultural counseling, the impact of culture, ethnicity, race,
    religion, sexual orientation, gender identity, disability,
    socioeconomic status, and other diversity factors on help-seeking,
    treatment engagement, and recovery. Understanding cultural humility
    as a lifelong process of self-reflection, learning, and redressing
    power imbalances.
  \item
    \textbf{Classification:} Knowledge
  \item
    \textbf{Specificity:} General (essential for all counseling)
  \item
    \textbf{O*NET:} 21-1011.00, 21-1018.00, 21-1015.00
  \item
    \textbf{Stability/Malleability:} Developable
  \item
    \textbf{Explicit/Tacit:} Explicit (principles), Tacit (humility
    development)
  \item
    \textbf{Prerequisites:} O1 (Self-Awareness). Foundation for
    culturally responsive practice.
  \end{itemize}
\item
  \textbf{KSAO Name:} Skill in Culturally Responsive and Inclusive
  Practices

  \begin{itemize}
  \tightlist
  \item
    \textbf{Description:} Ability to apply multicultural perspectives
    and cultural humility in all aspects of counseling, adapting
    communication and interventions to be congruent with clients'
    cultural backgrounds, values, and beliefs. This includes using
    non-stigmatizing language, identifying, responding to, and
    advocating for diversity, equity, and inclusion (DEI) in care, and
    addressing systemic barriers.
  \item
    \textbf{Classification:} Skill
  \item
    \textbf{Specificity:} General (essential for all counseling) but
    with specific SUD considerations.
  \item
    \textbf{O*NET:} 21-1011.00, 21-1018.00, 21-1015.00
  \item
    \textbf{Stability/Malleability:} Developable (requires ongoing
    effort and self-reflection)
  \item
    \textbf{Explicit/Tacit:} Both (knowledge is explicit, application is
    nuanced and tacit)
  \item
    \textbf{Prerequisites:} KSAO 55, O2 (Cultural Humility), O6
    (Empathy).
  \end{itemize}
\item
  \textbf{KSAO Name:} Knowledge of Confidentiality, Privacy Laws, and
  Informed Consent

  \begin{itemize}
  \tightlist
  \item
    \textbf{Description:} Comprehensive understanding of federal and
    state laws/regulations governing client confidentiality and privacy
    in SUD treatment (e.g., HIPAA, 42 CFR Part 2). Understanding
    ethical/legal requirements for obtaining informed consent, including
    disclosure of services, risks/benefits, alternatives,
    confidentiality limits, and client rights.
  \item
    \textbf{Classification:} Knowledge
  \item
    \textbf{Specificity:} Specialized (42 CFR Part 2 is highly specific
    to SUD)
  \item
    \textbf{O*NET:} 21-1011.00, 21-1018.00
  \item
    \textbf{Stability/Malleability:} Developable (laws are updated)
  \item
    \textbf{Explicit/Tacit:} Explicit
  \item
    \textbf{Prerequisites:} Basic understanding of legal principles.
  \end{itemize}
\item
  \textbf{KSAO Name:} Skill in Applying Confidentiality, Privacy, and
  Informed Consent Procedures

  \begin{itemize}
  \tightlist
  \item
    \textbf{Description:} Ability to consistently adhere to
    confidentiality/privacy laws in all aspects of practice. Proficiency
    in clearly explaining necessary information to clients, ensuring
    their understanding, and obtaining/documenting voluntary informed
    consent.
  \item
    \textbf{Classification:} Skill
  \item
    \textbf{Specificity:} Specialized
  \item
    \textbf{O*NET:} 21-1011.00, 21-1018.00
  \item
    \textbf{Stability/Malleability:} Developable
  \item
    \textbf{Explicit/Tacit:} Both (procedures are explicit,
    communication is nuanced)
  \item
    \textbf{Prerequisites:} KSAO 57, KSAO 46 (Therapeutic
    Communication).
  \end{itemize}
\item
  \textbf{KSAO Name:} Knowledge of Telehealth Legal, Ethical, and
  Technological Standards

  \begin{itemize}
  \tightlist
  \item
    \textbf{Description:} Understanding the specific legal (licensure,
    scope of practice across jurisdictions), ethical (privacy, security,
    informed consent, boundaries, equity of access), and technological
    considerations (platform choice, troubleshooting) relevant to
    delivering SUD services via telehealth.
  \item
    \textbf{Classification:} Knowledge
  \item
    \textbf{Specificity:} Specialized
  \item
    \textbf{O*NET:} 21-1011.00, 21-1018.00
  \item
    \textbf{Stability/Malleability:} Developable (rapidly evolving
    field)
  \item
    \textbf{Explicit/Tacit:} Explicit
  \item
    \textbf{Prerequisites:} KSAO 53, KSAO 57.
  \end{itemize}
\item
  \textbf{KSAO Name:} Knowledge of Client Rights and Grievance Processes

  \begin{itemize}
  \tightlist
  \item
    \textbf{Description:} Understanding of fundamental client/patient
    rights in SUD treatment and established procedures for clients to
    file grievances, and how counselors should respond.
  \item
    \textbf{Classification:} Knowledge
  \item
    \textbf{Specificity:} Specialized (agency/system specific for
    grievances)
  \item
    \textbf{O*NET:} 21-1011.00, 21-1018.00
  \item
    \textbf{Stability/Malleability:} Developable
  \item
    \textbf{Explicit/Tacit:} Explicit
  \item
    \textbf{Prerequisites:} Ethical and legal knowledge.
  \end{itemize}
\item
  \textbf{KSAO Name:} Skill in Upholding Client Rights and Responding to
  Grievances

  \begin{itemize}
  \tightlist
  \item
    \textbf{Description:} Ability to consistently respect, protect, and
    advocate for client rights. Ability to address client grievances
    professionally, ethically, and timely, following established
    procedures.
  \item
    \textbf{Classification:} Skill
  \item
    \textbf{Specificity:} General (core ethical practice)
  \item
    \textbf{O*NET:} 21-1011.00, 21-1018.00
  \item
    \textbf{Stability/Malleability:} Developable
  \item
    \textbf{Explicit/Tacit:} Both
  \item
    \textbf{Prerequisites:} KSAO 60, O3 (Ethical Integrity).
  \end{itemize}
\item
  \textbf{KSAO Name:} Knowledge of Clinical Supervision and Consultation
  Processes

  \begin{itemize}
  \tightlist
  \item
    \textbf{Description:} Understanding the purpose, benefits, models,
    and processes of engaging in clinical supervision and consultation
    for professional development, ethical decision-making, quality
    client care, and management of secondary trauma.
  \item
    \textbf{Classification:} Knowledge
  \item
    \textbf{Specificity:} General (professional development)
  \item
    \textbf{O*NET:} 21-1011.00, 21-1018.00
  \item
    \textbf{Stability/Malleability:} Developable
  \item
    \textbf{Explicit/Tacit:} Explicit
  \item
    \textbf{Prerequisites:} Understanding of professional roles.
  \end{itemize}
\item
  \textbf{KSAO Name:} Skill in Utilizing Clinical Supervision and
  Consultation

  \begin{itemize}
  \tightlist
  \item
    \textbf{Description:} Ability to proactively seek, prepare for, and
    effectively use clinical supervision and consultation to enhance
    counseling skills, address challenging cases, ensure ethical
    practice, and manage personal/professional impact of the work.
  \item
    \textbf{Classification:} Skill
  \item
    \textbf{Specificity:} General (professional development skill)
  \item
    \textbf{O*NET:} 21-1011.00, 21-1018.00
  \item
    \textbf{Stability/Malleability:} Developable
  \item
    \textbf{Explicit/Tacit:} Both (process is explicit, effective
    engagement is tacit)
  \item
    \textbf{Prerequisites:} KSAO 62, O1 (Self-Awareness).
  \end{itemize}
\item
  \textbf{KSAO Name:} Knowledge of Secondary Trauma, Burnout, and
  Self-Care Strategies

  \begin{itemize}
  \tightlist
  \item
    \textbf{Description:} Understanding the concepts of secondary
    traumatic stress (STS), compassion fatigue, burnout, and vicarious
    traumatization, including their risks, symptoms, and a range of
    evidence-informed self-care strategies (e.g., mindfulness, peer
    support, work-life balance, seeking therapy).
  \item
    \textbf{Classification:} Knowledge
  \item
    \textbf{Specificity:} General (for helping professions) but crucial
    for SUD.
  \item
    \textbf{O*NET:} 21-1011.00, 21-1018.00
  \item
    \textbf{Stability/Malleability:} Developable
  \item
    \textbf{Explicit/Tacit:} Explicit
  \item
    \textbf{Prerequisites:} O1 (Self-Awareness).
  \end{itemize}
\item
  \textbf{KSAO Name:} Skill in Professional Self-Care and Burnout
  Prevention

  \begin{itemize}
  \tightlist
  \item
    \textbf{Description:} The ability to actively and consistently
    engage in practices that promote mental, emotional, physical, and
    spiritual well-being to mitigate the risks of secondary trauma and
    burnout, and to seek support when needed.
  \item
    \textbf{Classification:} Skill
  \item
    \textbf{Specificity:} General (for helping professions)
  \item
    \textbf{O*NET:} 21-1011.00, 21-1018.00
  \item
    \textbf{Stability/Malleability:} Developable (requires ongoing
    commitment and practice)
  \item
    \textbf{Explicit/Tacit:} Both (strategies are explicit, consistent
    application is more tacit/habitual)
  \item
    \textbf{Prerequisites:} KSAO 64, O7 (Commitment to Professional
    Self-Care).
  \end{itemize}
\end{enumerate}

\subsubsection{Cross-Cutting Abilities
(A)}\label{cross-cutting-abilities-a}

These are more general, enduring capacities that underpin successful
performance across domains.

A1. \textbf{KSAO Name:} Ability to Synthesize and Integrate Complex
Information * \textbf{Description:} Capacity to integrate diverse and
complex information (e.g., neurobiological, psychological, social,
cultural, historical, and personal factors) into a cohesive
understanding of SUDs and individual client presentations for
assessment, diagnosis, and treatment planning. *
\textbf{Classification:} Ability * \textbf{Specificity:} General *
\textbf{O*NET:} 21-1011.00, 21-1018.00 *
\textbf{Stability/Malleability:} Developable (can be improved with
practice and critical thinking) * \textbf{Explicit/Tacit:} Tacit (how
one processes and connects information) * \textbf{Prerequisites:}
Foundational knowledge KSAOs, A2 (Critical Thinking).

A2. \textbf{KSAO Name:} Ability for Critical Thinking and Clinical
Reasoning * \textbf{Description:} Capacity to analyze different models,
theories, assessment data, and research findings related to SUDs,
considering their strengths, limitations, and implications for practice.
Ability to engage in sound clinical reasoning to make informed decisions
about client care. * \textbf{Classification:} Ability *
\textbf{Specificity:} General * \textbf{O*NET:} 21-1011.00, 21-1018.00 *
\textbf{Stability/Malleability:} Developable * \textbf{Explicit/Tacit:}
Both (critical thinking skills can be taught, application becomes more
tacit) * \textbf{Prerequisites:} Foundational knowledge, logical
reasoning skills.

A3. \textbf{KSAO Name:} Ability to Individualize Treatment Approaches *
\textbf{Description:} Capacity to tailor interventions and support
strategies to meet the unique needs, preferences, strengths, cultural
background, stage of change, and circumstances of each client, rather
than applying a one-size-fits-all approach. * \textbf{Classification:}
Ability * \textbf{Specificity:} General (core to effective counseling) *
\textbf{O*NET:} 21-1011.00, 21-1018.00 *
\textbf{Stability/Malleability:} Developable * \textbf{Explicit/Tacit:}
Tacit (involves clinical judgment, creativity, and empathy) *
\textbf{Prerequisites:} Comprehensive knowledge and skills KSAOs, O6
(Empathy).

A4. \textbf{KSAO Name:} Ability to Engage Clients Virtually *
\textbf{Description:} The capacity to establish and maintain a
therapeutic connection, effectively communicate, and facilitate
therapeutic processes with clients when services are delivered through a
telehealth modality. * \textbf{Classification:} Ability *
\textbf{Specificity:} Specialized (in the context of telehealth) *
\textbf{O*NET:} 21-1011.00, 21-1018.00 *
\textbf{Stability/Malleability:} Developable (can be enhanced with
training and experience) * \textbf{Explicit/Tacit:} Tacit (though
strategies can be taught explicitly) * \textbf{Prerequisites:} KSAO 46
(Therapeutic Communication), interpersonal skills.

A5. \textbf{KSAO Name:} Ability for Technological Adaptability *
\textbf{Description:} The capacity to learn and effectively use chosen
telehealth technologies, electronic health records, and other relevant
software or devices necessary for service delivery, documentation, and
professional practice. * \textbf{Classification:} Ability *
\textbf{Specificity:} General (in modern workplace) but applied to SUD
counseling context. * \textbf{O*NET:} 21-1011.00, 21-1018.00 *
\textbf{Stability/Malleability:} Developable * \textbf{Explicit/Tacit:}
Tacit (learning new tech often involves trial and error beyond explicit
instructions) * \textbf{Prerequisites:} Basic comfort with technology,
O8 (Positive Attitude Towards Telehealth).

\subsubsection{Cross-Cutting Other Characteristics
(O)}\label{cross-cutting-other-characteristics-o}

These are traits, values, or orientations that influence professional
behavior and effectiveness.

O1. \textbf{KSAO Name:} Self-Awareness (Professional Context) *
\textbf{Description:} The capacity for introspection and understanding
of one's own beliefs, values, biases, emotional reactions, cultural
worldview, and professional limitations, and how these may impact the
therapeutic process, particularly concerning boundaries,
self-disclosure, countertransference, and vicarious trauma. *
\textbf{Classification:} Other Characteristic * \textbf{Specificity:}
General (essential for all helping professions) * \textbf{O*NET:}
21-1011.00, 21-1018.00, 21-1015.00 * \textbf{Stability/Malleability:}
Developable (through reflection, supervision, experience) *
\textbf{Explicit/Tacit:} Tacit (developed through experience and
reflection, though concepts can be taught explicitly) *
\textbf{Prerequisites:} Foundational for ethical practice, cultural
humility, and personal well-being.

O2. \textbf{KSAO Name:} Cultural Humility * \textbf{Description:} A
lifelong commitment to self-evaluation and self-critique, to redressing
power imbalances in the therapeutic relationship, and to developing
mutually respectful partnerships with communities and individuals from
diverse cultural backgrounds. Involves recognizing that the counselor is
not the expert on the client's cultural experience. *
\textbf{Classification:} Other Characteristic * \textbf{Specificity:}
General (essential for ethical and effective practice) * \textbf{O*NET:}
21-1011.00, 21-1018.00 * \textbf{Stability/Malleability:} Developable
(an ongoing process of learning and reflection) *
\textbf{Explicit/Tacit:} Both (concept is explicit, embodiment is tacit
and ongoing) * \textbf{Prerequisites:} O1 (Self-Awareness), openness to
learning.

O3. \textbf{KSAO Name:} Ethical Integrity and Commitment to Ethical
Practice * \textbf{Description:} A consistent adherence to strong moral
and ethical principles and values in professional conduct, including a
commitment to non-maleficence, beneficence, autonomy, justice, and
fidelity. This includes a commitment to non-stigmatizing practice. *
\textbf{Classification:} Other Characteristic * \textbf{Specificity:}
General (core for all professions, esp.~helping professions) *
\textbf{O*NET:} 21-1011.00, 21-1018.00, 21-1015.00 *
\textbf{Stability/Malleability:} Developable/Reinforceable (through
education, commitment, and reflection), but also reflects character. *
\textbf{Explicit/Tacit:} Tacit (demonstrated through behavior), though
ethical principles are explicit. * \textbf{Prerequisites:} Moral
reasoning, KSAO 53 (Knowledge of Professional Ethics).

O4. \textbf{KSAO Name:} Recovery-Oriented Mindset *
\textbf{Description:} A fundamental belief in the potential for
individuals to achieve self-defined recovery from SUDs and a commitment
to supporting person-centered, strengths-based, holistic, and hopeful
recovery journeys, recognizing multiple pathways. *
\textbf{Classification:} Other Characteristic * \textbf{Specificity:}
Specialized (though principles are spreading in behavioral health) *
\textbf{O*NET:} 21-1011.00, 21-1018.00 *
\textbf{Stability/Malleability:} Developable (through education,
experience, and exposure to recovery stories) * \textbf{Explicit/Tacit:}
Both (principles are explicit, internal adoption and conveyance are
tacit) * \textbf{Prerequisites:} KSAO 7 (Knowledge of Recovery
Principles).

O5. \textbf{KSAO Name:} Commitment to Lifelong Learning and Professional
Development * \textbf{Description:} Recognition that the field of SUDs
is constantly evolving (new research, substances, treatment approaches,
terminology, policies) and a proactive dedication to continuous
learning, skill enhancement, and staying current with EBPs to maintain
professional competence and improve client outcomes. *
\textbf{Classification:} Other Characteristic (Professionalism,
Motivation) * \textbf{Specificity:} General (expected in most
professions) * \textbf{O*NET:} 21-1011.00, 21-1018.00 *
\textbf{Stability/Malleability:} Developable (can be cultivated and is
often required for licensure/certification) * \textbf{Explicit/Tacit:}
Both (explicit expectation via CEUs, tacit drive to learn and improve) *
\textbf{Prerequisites:} Intrinsic motivation, awareness of field
dynamics.

O6. \textbf{KSAO Name:} Empathy * \textbf{Description:} The ability to
understand, accurately perceive, and sensitively respond to the client's
feelings, thoughts, and experiences from their frame of reference, and
to communicate that understanding effectively. *
\textbf{Classification:} Other Characteristic (with Ability components)
* \textbf{Specificity:} General (core for helping professions) *
\textbf{O*NET:} 21-1011.00, 21-1018.00, 21-1015.00 *
\textbf{Stability/Malleability:} Can be developed/enhanced through
training and self-reflection, though some innate capacity may exist. *
\textbf{Explicit/Tacit:} Tacit (experienced and expressed), though
components (e.g., reflective listening) can be explicitly taught. *
\textbf{Prerequisites:} O1 (Self-Awareness). Foundational for
therapeutic alliance and effective communication.

O7. \textbf{KSAO Name:} Commitment to Professional Self-Care *
\textbf{Description:} An ongoing dedication to monitoring one's own
well-being and actively engaging in practices that prevent or mitigate
compassion fatigue, secondary trauma, and burnout, thereby maintaining
capacity for ethical and effective practice. * \textbf{Classification:}
Other Characteristic (Attitude, Value) * \textbf{Specificity:} General
(for helping professions) * \textbf{O*NET:} 21-1011.00, 21-1018.00 *
\textbf{Stability/Malleability:} Developable (requires conscious effort
and can be learned) * \textbf{Explicit/Tacit:} Both (importance can be
explicitly taught, development as a habit is tacit) *
\textbf{Prerequisites:} KSAO 64 (Knowledge of Secondary
Trauma/Self-Care), O1 (Self-Awareness).

O8. \textbf{KSAO Name:} Positive Attitude Towards Telehealth (if
applicable) * \textbf{Description:} Possessing personal motivation and
favorable attitudes regarding the use and efficacy of telehealth as a
service delivery modality, when ethically and clinically appropriate. *
\textbf{Classification:} Other Characteristic (Attitude) *
\textbf{Specificity:} Specialized (relevant for telehealth providers) *
\textbf{O*NET:} 21-1011.00, 21-1018.00 *
\textbf{Stability/Malleability:} Developable (can be influenced by
experience, education, and outcomes) * \textbf{Explicit/Tacit:} Tacit
(though can be explicitly encouraged) * \textbf{Prerequisites:} Openness
to new technologies and service delivery models.

\begin{center}\rule{0.5\linewidth}{0.5pt}\end{center}

\subsubsection{Hierarchical Structure and Developmental Relationships
Among
KSAOs}\label{hierarchical-structure-and-developmental-relationships-among-ksaos}

\textbf{I. Dimensions and Sub-dimensions:}

The KSAO framework can be broadly organized into the following
interacting dimensions, largely aligning with the IC\&RC Performance
Domains, with the addition of overarching abilities and characteristics:

\begin{enumerate}
\def\labelenumi{\arabic{enumi}.}
\tightlist
\item
  \textbf{Dimension: Foundational Scientific and Contextual Knowledge
  (Domain 1 derived)}

  \begin{itemize}
  \tightlist
  \item
    \emph{Sub-dimensions (Examples):} KSAOs 1-15 (Knowledge of SUD
    Terminology, Neurobiology, Pharmacology, Co-occurring Conditions,
    Recovery Principles, Trauma, Stigma, SDOH, Harm Reduction, Treatment
    Systems, etc.)
  \end{itemize}
\item
  \textbf{Dimension: Clinical Assessment and Diagnostic Competencies
  (Domain 2 derived)}

  \begin{itemize}
  \tightlist
  \item
    \emph{Sub-dimensions (Examples):} KSAOs 16-26 (Knowledge and Skills
    in Interviewing, Screening/Assessment Instruments, Diagnostic
    Criteria Application, Case Conceptualization, Level of Care
    Determination, Clinical Decision-Making).
  \end{itemize}
\item
  \textbf{Dimension: Therapeutic Interventions and Service Delivery
  (Domain 3 derived)}

  \begin{itemize}
  \tightlist
  \item
    \emph{Sub-dimensions (Examples):} KSAOs 27-50 (Knowledge and Skills
    in EBPs, Treatment Planning, Relapse Prevention, Coping Skills,
    Group Counseling, Motivational Interviewing, Crisis Intervention,
    Discharge/Continuing Care, Referral/Care Coordination,
    Interprofessional Collaboration, Culturally Responsive Practice,
    Person-Centered/Trauma-Informed Approaches, Psychoeducation, Harm
    Reduction Application, Telehealth Delivery).
  \end{itemize}
\item
  \textbf{Dimension: Professional, Ethical, and Legal Practice (Domain 4
  derived)}

  \begin{itemize}
  \tightlist
  \item
    \emph{Sub-dimensions (Examples):} KSAOs 51-65 (Knowledge and Skills
    in Documentation, Ethics/Boundaries/Scope of Practice, Multicultural
    Counseling, Confidentiality/Informed Consent, Client
    Rights/Grievances, Supervision/Consultation, Self-Care/Burnout
    Prevention).
  \end{itemize}
\item
  \textbf{Dimension: Core Cognitive and Professional Abilities
  (Cross-Cutting Abilities A1-A5)}

  \begin{itemize}
  \tightlist
  \item
    These abilities (e.g., Synthesize Complex Info, Critical Thinking,
    Individualize Treatment, Virtual Engagement, Tech Adaptability)
    underpin performance across all other dimensions.
  \end{itemize}
\item
  \textbf{Dimension: Foundational Professional Characteristics
  (Cross-Cutting Other Characteristics O1-O8)}

  \begin{itemize}
  \tightlist
  \item
    These characteristics (e.g., Self-Awareness, Cultural Humility,
    Ethical Integrity, Recovery-Oriented Mindset, Lifelong Learning,
    Empathy, Commitment to Self-Care) shape the counselor's approach and
    effectiveness.
  \end{itemize}
\end{enumerate}

\textbf{II. Developmental Sequence:}

The acquisition and development of these KSAOs generally follow a
sequence:

\begin{enumerate}
\def\labelenumi{\arabic{enumi}.}
\tightlist
\item
  \textbf{Foundational Stage:}

  \begin{itemize}
  \tightlist
  \item
    Development of core \textbf{Other Characteristics} (O1-O8,
    especially Empathy, Ethical Integrity, Self-Awareness, Cultural
    Humility).
  \item
    Acquisition of fundamental \textbf{Knowledge} (Domain 1 KSAOs:
    Terminology, basic SUD concepts, Stigma, Recovery Principles).
  \item
    Development of basic \textbf{Abilities} like A2 (Critical Thinking).
  \end{itemize}
\item
  \textbf{Knowledge Expansion Stage:}

  \begin{itemize}
  \tightlist
  \item
    Deeper understanding of specialized \textbf{Knowledge} (Domain 1:
    Neurobiology, Pharmacology, Co-occurring conditions, specific models
    like SOC, Trauma-Informed Care, Harm Reduction; Domain 4: Ethics,
    Laws, Client Rights). This requires A1 (Ability to Synthesize
    Complex Info).
  \end{itemize}
\item
  \textbf{Core Clinical Skills Development Stage:}

  \begin{itemize}
  \tightlist
  \item
    Learning and practicing foundational \textbf{Skills} for client
    engagement (Domain 3: KSAO 46 - Therapeutic Communication/Rapport
    Building).
  \item
    Developing \textbf{Skills} in assessment and diagnosis (Domain 2:
    KSAO 17 - Clinical Interviewing; KSAO 19 - Using Assessment
    Instruments; KSAO 23 - Assessment, Diagnosis, Case
    Conceptualization; KSAO 26 - Clinical Decision-Making). This relies
    heavily on Domain 1 knowledge and A1/A2 abilities.
  \end{itemize}
\item
  \textbf{Intervention Skills Application Stage:}

  \begin{itemize}
  \tightlist
  \item
    Acquiring \textbf{Knowledge} of various EBPs and intervention
    strategies (Domain 3: KSAO 27 - Knowledge of EBPs, KSAO 31 - Relapse
    Prevention, KSAO 34 - Group Counseling, KSAO 36 - MI).
  \item
    Developing \textbf{Skills} in applying these interventions (Domain
    3: KSAO 28 - Applying EBPs, KSAO 30 - Treatment Planning, KSAO 33 -
    Teaching Coping Skills, KSAO 35 - Group Facilitation, KSAO 37 - MI
    Skills, KSAO 39 - Crisis Intervention, KSAO 47 -
    Person-Centered/Trauma-Informed Practices). This stage involves A3
    (Ability to Individualize Treatment).
  \end{itemize}
\item
  \textbf{Professional Practice and Systems Navigation Stage:}

  \begin{itemize}
  \tightlist
  \item
    Mastering \textbf{Skills} related to documentation, referrals,
    collaboration, and ethical practice in complex situations (Domain 3:
    KSAO 43 - Referrals/Care Coordination, KSAO 44 - Interprofessional
    Collaboration; Domain 4: KSAO 52 - Documentation, KSAO 54 -
    Boundaries/Scope, KSAO 56 - Culturally Responsive Practice, KSAO 58
    - Confidentiality/Informed Consent, KSAO 63 - Utilizing Supervision,
    KSAO 65 - Self-Care).
  \item
    If applicable, developing specialized skills like KSAO 50
    (Telehealth Delivery), supported by A4/A5 abilities and O8.
  \end{itemize}
\item
  \textbf{Continuous Refinement Stage:}

  \begin{itemize}
  \tightlist
  \item
    Ongoing commitment to O5 (Lifelong Learning), O1 (Self-Awareness),
    O2 (Cultural Humility), and O7 (Self-Care) to maintain and enhance
    competence throughout one's career.
  \end{itemize}
\end{enumerate}

\textbf{III. KSAOs Serving as Foundations for Others (Examples):}

\begin{itemize}
\tightlist
\item
  \textbf{O1 (Self-Awareness), O3 (Ethical Integrity), O6 (Empathy):}
  Foundational for almost all client-facing skills and professional
  conduct.
\item
  \textbf{KSAO 1 (Knowledge of SUD Terminology):} Prerequisite for
  understanding nearly all other SUD-specific knowledge and for
  effective communication.
\item
  \textbf{Domain 1 KSAOs (Scientific Knowledge):} Collectively provide
  the theoretical and empirical basis for assessment (Domain 2) and
  treatment (Domain 3) KSAOs.
\item
  \textbf{KSAO 46 (Skill in Therapeutic Communication and Rapport
  Building):} Essential foundation for effective interviewing,
  assessment, all counseling interventions, and collaboration.
\item
  \textbf{KSAO 23 (Skill in SUD Assessment, Diagnosis, and Case
  Conceptualization):} Prerequisite for effective KSAO 30 (Treatment
  Planning) and selection of appropriate interventions.
\item
  \textbf{KSAO 53 (Knowledge of Professional Ethics, Boundaries, and
  Scope of Practice):} Underpins all professional actions and skills in
  Domain 4.
\item
  \textbf{KSAO 55 (Knowledge of Multicultural Counseling/Cultural
  Humility):} Foundational for KSAO 56 (Skill in Culturally Responsive
  Practice) and influences the application of all other clinical skills.
\end{itemize}

This comprehensive framework provides a robust map of the KSAOs required
for competent SUD counseling, based on the integration of the provided
analyses. It highlights the multifaceted nature of the profession,
blending deep knowledge, refined skills, critical abilities, and
essential personal characteristics.

\section{Hierarchical Structure and Developmental
Relationships}\label{hierarchical-structure-and-developmental-relationships}

This section presents the hierarchical organization of KSAOs, showing
how they relate to each other in terms of:

\begin{enumerate}
\def\labelenumi{\arabic{enumi}.}
\tightlist
\item
  \textbf{Dimensions vs.~Sub-dimensions}: Major competency domains and
  their components
\item
  \textbf{Developmental Sequences}: The logical progression for
  acquiring competencies
\item
  \textbf{Foundational Relationships}: Which KSAOs serve as
  prerequisites for others
\end{enumerate}

\subsubsection{Hierarchical Structure and Developmental Relationships
Among
KSAOs}\label{hierarchical-structure-and-developmental-relationships-among-ksaos-1}

\textbf{I. Dimensions and Sub-dimensions:}

The KSAO framework can be broadly organized into the following
interacting dimensions, largely aligning with the IC\&RC Performance
Domains, with the addition of overarching abilities and characteristics:

\begin{enumerate}
\def\labelenumi{\arabic{enumi}.}
\tightlist
\item
  \textbf{Dimension: Foundational Scientific and Contextual Knowledge
  (Domain 1 derived)}

  \begin{itemize}
  \tightlist
  \item
    \emph{Sub-dimensions (Examples):} KSAOs 1-15 (Knowledge of SUD
    Terminology, Neurobiology, Pharmacology, Co-occurring Conditions,
    Recovery Principles, Trauma, Stigma, SDOH, Harm Reduction, Treatment
    Systems, etc.)
  \end{itemize}
\item
  \textbf{Dimension: Clinical Assessment and Diagnostic Competencies
  (Domain 2 derived)}

  \begin{itemize}
  \tightlist
  \item
    \emph{Sub-dimensions (Examples):} KSAOs 16-26 (Knowledge and Skills
    in Interviewing, Screening/Assessment Instruments, Diagnostic
    Criteria Application, Case Conceptualization, Level of Care
    Determination, Clinical Decision-Making).
  \end{itemize}
\item
  \textbf{Dimension: Therapeutic Interventions and Service Delivery
  (Domain 3 derived)}

  \begin{itemize}
  \tightlist
  \item
    \emph{Sub-dimensions (Examples):} KSAOs 27-50 (Knowledge and Skills
    in EBPs, Treatment Planning, Relapse Prevention, Coping Skills,
    Group Counseling, Motivational Interviewing, Crisis Intervention,
    Discharge/Continuing Care, Referral/Care Coordination,
    Interprofessional Collaboration, Culturally Responsive Practice,
    Person-Centered/Trauma-Informed Approaches, Psychoeducation, Harm
    Reduction Application, Telehealth Delivery).
  \end{itemize}
\item
  \textbf{Dimension: Professional, Ethical, and Legal Practice (Domain 4
  derived)}

  \begin{itemize}
  \tightlist
  \item
    \emph{Sub-dimensions (Examples):} KSAOs 51-65 (Knowledge and Skills
    in Documentation, Ethics/Boundaries/Scope of Practice, Multicultural
    Counseling, Confidentiality/Informed Consent, Client
    Rights/Grievances, Supervision/Consultation, Self-Care/Burnout
    Prevention).
  \end{itemize}
\item
  \textbf{Dimension: Core Cognitive and Professional Abilities
  (Cross-Cutting Abilities A1-A5)}

  \begin{itemize}
  \tightlist
  \item
    These abilities (e.g., Synthesize Complex Info, Critical Thinking,
    Individualize Treatment, Virtual Engagement, Tech Adaptability)
    underpin performance across all other dimensions.
  \end{itemize}
\item
  \textbf{Dimension: Foundational Professional Characteristics
  (Cross-Cutting Other Characteristics O1-O8)}

  \begin{itemize}
  \tightlist
  \item
    These characteristics (e.g., Self-Awareness, Cultural Humility,
    Ethical Integrity, Recovery-Oriented Mindset, Lifelong Learning,
    Empathy, Commitment to Self-Care) shape the counselor's approach and
    effectiveness.
  \end{itemize}
\end{enumerate}

\textbf{II. Developmental Sequence:}

The acquisition and development of these KSAOs generally follow a
sequence:

\begin{enumerate}
\def\labelenumi{\arabic{enumi}.}
\tightlist
\item
  \textbf{Foundational Stage:}

  \begin{itemize}
  \tightlist
  \item
    Development of core \textbf{Other Characteristics} (O1-O8,
    especially Empathy, Ethical Integrity, Self-Awareness, Cultural
    Humility).
  \item
    Acquisition of fundamental \textbf{Knowledge} (Domain 1 KSAOs:
    Terminology, basic SUD concepts, Stigma, Recovery Principles).
  \item
    Development of basic \textbf{Abilities} like A2 (Critical Thinking).
  \end{itemize}
\item
  \textbf{Knowledge Expansion Stage:}

  \begin{itemize}
  \tightlist
  \item
    Deeper understanding of specialized \textbf{Knowledge} (Domain 1:
    Neurobiology, Pharmacology, Co-occurring conditions, specific models
    like SOC, Trauma-Informed Care, Harm Reduction; Domain 4: Ethics,
    Laws, Client Rights). This requires A1 (Ability to Synthesize
    Complex Info).
  \end{itemize}
\item
  \textbf{Core Clinical Skills Development Stage:}

  \begin{itemize}
  \tightlist
  \item
    Learning and practicing foundational \textbf{Skills} for client
    engagement (Domain 3: KSAO 46 - Therapeutic Communication/Rapport
    Building).
  \item
    Developing \textbf{Skills} in assessment and diagnosis (Domain 2:
    KSAO 17 - Clinical Interviewing; KSAO 19 - Using Assessment
    Instruments; KSAO 23 - Assessment, Diagnosis, Case
    Conceptualization; KSAO 26 - Clinical Decision-Making). This relies
    heavily on Domain 1 knowledge and A1/A2 abilities.
  \end{itemize}
\item
  \textbf{Intervention Skills Application Stage:}

  \begin{itemize}
  \tightlist
  \item
    Acquiring \textbf{Knowledge} of various EBPs and intervention
    strategies (Domain 3: KSAO 27 - Knowledge of EBPs, KSAO 31 - Relapse
    Prevention, KSAO 34 - Group Counseling, KSAO 36 - MI).
  \item
    Developing \textbf{Skills} in applying these interventions (Domain
    3: KSAO 28 - Applying EBPs, KSAO 30 - Treatment Planning, KSAO 33 -
    Teaching Coping Skills, KSAO 35 - Group Facilitation, KSAO 37 - MI
    Skills, KSAO 39 - Crisis Intervention, KSAO 47 -
    Person-Centered/Trauma-Informed Practices). This stage involves A3
    (Ability to Individualize Treatment).
  \end{itemize}
\item
  \textbf{Professional Practice and Systems Navigation Stage:}

  \begin{itemize}
  \tightlist
  \item
    Mastering \textbf{Skills} related to documentation, referrals,
    collaboration, and ethical practice in complex situations (Domain 3:
    KSAO 43 - Referrals/Care Coordination, KSAO 44 - Interprofessional
    Collaboration; Domain 4: KSAO 52 - Documentation, KSAO 54 -
    Boundaries/Scope, KSAO 56 - Culturally Responsive Practice, KSAO 58
    - Confidentiality/Informed Consent, KSAO 63 - Utilizing Supervision,
    KSAO 65 - Self-Care).
  \item
    If applicable, developing specialized skills like KSAO 50
    (Telehealth Delivery), supported by A4/A5 abilities and O8.
  \end{itemize}
\item
  \textbf{Continuous Refinement Stage:}

  \begin{itemize}
  \tightlist
  \item
    Ongoing commitment to O5 (Lifelong Learning), O1 (Self-Awareness),
    O2 (Cultural Humility), and O7 (Self-Care) to maintain and enhance
    competence throughout one's career.
  \end{itemize}
\end{enumerate}

\textbf{III. KSAOs Serving as Foundations for Others (Examples):}

\begin{itemize}
\tightlist
\item
  \textbf{O1 (Self-Awareness), O3 (Ethical Integrity), O6 (Empathy):}
  Foundational for almost all client-facing skills and professional
  conduct.
\item
  \textbf{KSAO 1 (Knowledge of SUD Terminology):} Prerequisite for
  understanding nearly all other SUD-specific knowledge and for
  effective communication.
\item
  \textbf{Domain 1 KSAOs (Scientific Knowledge):} Collectively provide
  the theoretical and empirical basis for assessment (Domain 2) and
  treatment (Domain 3) KSAOs.
\item
  \textbf{KSAO 46 (Skill in Therapeutic Communication and Rapport
  Building):} Essential foundation for effective interviewing,
  assessment, all counseling interventions, and collaboration.
\item
  \textbf{KSAO 23 (Skill in SUD Assessment, Diagnosis, and Case
  Conceptualization):} Prerequisite for effective KSAO 30 (Treatment
  Planning) and selection of appropriate interventions.
\item
  \textbf{KSAO 53 (Knowledge of Professional Ethics, Boundaries, and
  Scope of Practice):} Underpins all professional actions and skills in
  Domain 4.
\item
  \textbf{KSAO 55 (Knowledge of Multicultural Counseling/Cultural
  Humility):} Foundational for KSAO 56 (Skill in Culturally Responsive
  Practice) and influences the application of all other clinical skills.
\end{itemize}

This comprehensive framework provides a robust map of the KSAOs required
for competent SUD counseling, based on the integration of the provided
analyses. It highlights the multifaceted nature of the profession,
blending deep knowledge, refined skills, critical abilities, and
essential personal characteristics.

\section{Applications for Workforce
Development}\label{applications-for-workforce-development}

The integrated KSAO framework presented in this document can be applied
in multiple ways to enhance workforce development in the field of
substance use disorder counseling:

\begin{enumerate}
\def\labelenumi{\arabic{enumi}.}
\item
  \textbf{Curriculum Design}: Educational institutions can use this
  framework to design comprehensive training programs that
  systematically address all required competencies.
\item
  \textbf{Certification Standards}: Certification bodies can align
  requirements with evidence-based KSAOs rather than solely focusing on
  knowledge or hours of training.
\item
  \textbf{Professional Development}: Practitioners can identify gaps in
  their skill set and target specific areas for improvement.
\item
  \textbf{Supervision Focus}: Clinical supervisors can use the framework
  to provide structured guidance for developing counselors.
\item
  \textbf{Assessment Tools}: The framework provides a foundation for
  developing assessment instruments to measure counselor competency.
\end{enumerate}

\section{Conclusion}\label{conclusion}

This integrated KSAO framework represents a significant advancement in
understanding the competencies required for effective substance use
disorder counseling. By systematically extracting and organizing KSAOs
from authoritative textbook material, this project demonstrates how
AI-assisted analysis can support evidence-based workforce development.

The framework highlights the complex, multifaceted nature of SUD
counseling, requiring a diverse set of knowledge areas, clinical and
interpersonal skills, cognitive abilities, and professional
characteristics. It also reveals the developmental nature of these
competencies, with clear progression pathways and foundational
relationships.

Future work should extend this analysis to the remaining textbook
chapters to create a truly comprehensive framework, followed by
validation studies with subject matter experts and practicing
counselors.




\end{document}
